Dandelions are known as constructive medical raw materials, which also support the economy. However, some regard dandelions as invasive species due to their ability to endure the harsh environment and fast speed of propagation. Based on that circumstance, this paper delves into the investigation of dandelion population prediction as well as its impact factors on the risk index.

In the first section, we develop a differential equation model to describe the behavior of the dandelion population. We first divide the population into two main groups: seeds and puffballs. Considering their respective properties, we separate them into two different differential equations. In the next step, we gave each of the parameters in the differential equation a concise and reasonable formula. In addition, we have taken resource competitionabs into consideration, which further completely describes their behaviors. By evaluating growth rate, reproduction rate, death rate as well and resource competition, we finally derived around $160$ dandelions in a one-hectare area.

In the next section, we used the Cellular Automaton model to simulate the behavior of dandelions from a concrete perspective. We divided the field into a lattice made up of tons of square cells, with a side length of $\SI{0.01}{m}$. In the lattices, each cell either contains a dandelion or not. For those containing a dandelion, the dandelion will interact with the resource level of cells around it. The interaction can better describe the individuality of the dandelions, the long-term interaction, the regional differences, and the continuity of the life span. Via doing this, the result can generate a vivid visualization of the field, and give us more information about the periodic behaviors and inter-plant influences.

In response to question 2, we construct an evaluation model to assess the invasiveness of dandelions. By introducing a list of differential equations, we successfully analyze the interaction between two species. Instead of solely focusing on the contributions of different factors to the impact factor, we take an indirect step, concentrating on the parameters in the differential equations. Then we incorporate the AHP model to further outline the importance of various factors to the parameters, and then to the impact factor. With that thought, we ultimately conclude that dandelions should be considered an invasive species, as well as Pueraria montana and tumbleweeds, the two publicly acknowledged invasive species.